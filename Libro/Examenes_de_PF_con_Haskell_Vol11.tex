% Examenes_de_PF_con_Haskell_Vol11.tex
% Exámenes de programación funcional con Haskell (Curso 2019-20)
% José A. Alonso Jiménez <jalonso@us.es>
% Sevilla, 25 de noviembre de 2019
% =============================================================================

\documentclass[a4paper,12pt,twoside]{book}

%%%%%%%%%%%%%%%%%%%%%%%%%%%%%%%%%%%%%%%%%%%%%%%%%%%%%%%%%%%%%%%%%%%%%%%%%%%%%%
%% § Paquetes adicionales                                                   %%
%%%%%%%%%%%%%%%%%%%%%%%%%%%%%%%%%%%%%%%%%%%%%%%%%%%%%%%%%%%%%%%%%%%%%%%%%%%%%%

% Configuración para XeLaTeX
\usepackage{fontspec}
\usepackage{xltxtra}
\defaultfontfeatures{Ligatures=TeX,Numbers=OldStyle}
\setromanfont{DejaVu Sans}
% \setsansfont{Arial}
\setmonofont{DejaVu Sans Mono}[Scale={0.90}]

% Notas: La lista de fuentes disponibles se obtiene con fc-list

% \usepackage{ucs}
% \usepackage[utf8]{inputenc}        % Acentos de UTF8
\usepackage[spanish]{babel}        % Castellanización.
% \usepackage[T1]{fontenc}           % Codificación T1 con European Computer
%                                    % Modern.  
% \usepackage{graphicx}
\usepackage{fancyvrb}              % Verbatim extendido
% \usepackage{mathpazo}              % Fuentes semejante a palatino
% \usepackage[scaled=.90]{helvet}
% \usepackage{cmtt}
% \renewcommand{\ttdefault}{cmtt}
\usepackage{a4wide}
\usepackage{minted}

\linespread{1.05}                  % Distancia entre líneas
\setlength{\parindent}{2em}        % Indentación de comienzo de párrafo
% \deactivatetilden                  % Elima uso de ~ para la eñe
\raggedbottom                      % No ajusta los espacios verticales.

\usepackage[%
  colorlinks=true,
  urlcolor=blue,
  % pdftex,
  pdfauthor={José A. Alonso <jalonso@us.es>},%
  pdftitle={Examenes de Informatica de 1 de Matematicas},%
  pdfstartview=FitH,%
  bookmarks=false]{hyperref}      

\setcounter{tocdepth}{1}
\setcounter{secnumdepth}{4}

\usepackage{tocstyle}
\usetocstyle{KOMAlike}

\newenvironment{enumerate*}%
  {\vspace*{-0mm}
   \begin{enumerate}%
    \setlength{\itemsep}{0pt}%
    \setlength{\parskip}{0pt}}%
  {\vspace*{-0mm}
   \end{enumerate}}

\newenvironment{itemize*}%
  {\vspace*{-0mm}
   \begin{itemize}%
    \setlength{\itemsep}{0pt}%
    \setlength{\parskip}{0pt}}%
  {\vspace*{-0mm}
   \end{itemize}}

%%%%%%%%%%%%%%%%%%%%%%%%%%%%%%%%%%%%%%%%%%%%%%%%%%%%%%%%%%%%%%%%%%%%%%%%%%%%%%
%% § Cabeceras                                                              %%
%%%%%%%%%%%%%%%%%%%%%%%%%%%%%%%%%%%%%%%%%%%%%%%%%%%%%%%%%%%%%%%%%%%%%%%%%%%%%%

\usepackage{fancyhdr}

\addtolength{\headheight}{\baselineskip}

\pagestyle{fancy}

\cfoot{}

\fancyhead{}
\fancyhead[RE]{\mdseries\sffamily \nouppercase{\leftmark}}
\fancyhead[LO]{\mdseries\sffamily \nouppercase{\rightmark}}
\fancyhead[LE,RO]{\mdseries\sffamily \thepage}

%%%%%%%%%%%%%%%%%%%%%%%%%%%%%%%%%%%%%%%%%%%%%%%%%%%%%%%%%%%%%%%%%%%%%%%%%%%%%%
%% § Definiciones                                                           %%
%%%%%%%%%%%%%%%%%%%%%%%%%%%%%%%%%%%%%%%%%%%%%%%%%%%%%%%%%%%%%%%%%%%%%%%%%%%%%%

\input definiciones

%%%%%%%%%%%%%%%%%%%%%%%%%%%%%%%%%%%%%%%%%%%%%%%%%%%%%%%%%%%%%%%%%%%%%%%%%%%%%%
%% § Título                                                                 %%
%%%%%%%%%%%%%%%%%%%%%%%%%%%%%%%%%%%%%%%%%%%%%%%%%%%%%%%%%%%%%%%%%%%%%%%%%%%%%%

\title{
  {\LARGE Exámenes de \\ ``Programaci\'on funcional con Haskell''} \\ 
  {\large Vol. 11 (Curso 2019--20)}} 
\author{
  \href{http://www.cs.us.es/~jalonso}{José A. Alonso Jiménez}}
\date{\vfill \hrule \vspace*{2mm}
  \begin{tabular}{l}
      \href{http://www.cs.us.es/glc}
           {Grupo de Lógica Computacional} \\
      \href{http://www.cs.us.es}
           {Dpto. de Ciencias de la Computación e Inteligencia Artificial} \\
      \href{http://www.us.es}
           {Universidad de Sevilla}  \\
      Sevilla, 25 de noviembre de 2019
  \end{tabular}\hfill\mbox{}}

%%%%%%%%%%%%%%%%%%%%%%%%%%%%%%%%%%%%%%%%%%%%%%%%%%%%%%%%%%%%%%%%%%%%%%%%%%%%%%
%% § Documento                                                              %%
%%%%%%%%%%%%%%%%%%%%%%%%%%%%%%%%%%%%%%%%%%%%%%%%%%%%%%%%%%%%%%%%%%%%%%%%%%%%%%

% \includeonly{}

% \includexmp{licencia}

\begin{document}

\maketitle
\newpage

\input{Licencia/licenciaCC}
\newpage

\tableofcontents
\clearpage

\renewcommand{\chaptername}{}

\chapter*{Introducción}
\addcontentsline{toc}{chapter}{Introducción}

Este libro es una recopilación de las soluciones de
ejercicios de los exámenes de programación funcional con Haskell de la
\href{http://www.cs.us.es/~jalonso/cursos/i1m-19}
     {asignatura de Informática (curso 2019--20)}
del
\href{http://www.matematicas.us.es/estudios/grado-en-matematicas}
     {Grado en Matemática} 
de la 
\href{http://www.us.es/}
     {Universidad de Sevilla}.

Los exámenes se realizaron en el aula de informática y su duración
fue de 2 horas. La materia de cada examen es la impartida desde el
comienzo del curso (generalmente, el 1 de octubre) hasta la fecha
del examen. Dicha materia se encuentra en los libros de temas y
ejercicios del curso:
\begin{itemize}
\item
  \href{https://www.cs.us.es/~jalonso/cursos/i1m-19/temas/2019-20-IM-temas-PF.pdf}
  {Temas de programación funcional (curso 2019--20)}\
  \footnote{\url{https://www.cs.us.es/~jalonso/cursos/i1m-19/temas/2019-20-IM-temas-PF.pdf}} 
\item
  \href{https://www.cs.us.es/~jalonso/cursos/i1m-19/ejercicios/ejercicios-I1M-2019.pdf}
  {Ejercicios de ``Informática de 1º de Matemáticas'' (2019--20)}\
  \footnote{\url{https://www.cs.us.es/~jalonso/cursos/i1m-19/ejercicios/ejercicios-I1M-2019.pdf}}
\item
  \href{http://www.cs.us.es/~jalonso/publicaciones/Piensa_en_Haskell.pdf}
  {Piensa en Haskell (Ejercicios de programación funcional con Haskell)}\
  \footnote{\url{http://www.cs.us.es/~jalonso/publicaciones/Piensa_en_Haskell.pdf}}
\end{itemize}

El libro consta de 5 capítulos correspondientes a 5 grupos de la
asignatura. En cada capítulo hay una sección por cada uno de los
exámenes del grupo. Los ejercicios de cada examen han sido propuestos
por los profesores de su grupo (cuyos nombres aparecen en el título del
capítulo). Sin embargo, los he modificado para unificar el estilo de su
presentación.

Finalmente, el libro contiene dos apéndices. Uno con el método de Polya
de resolución de problemas (sobre el que se hace énfasis durante todo el
curso) y el otro con un resumen de las funciones de Haskell de uso más
frecuente.

Los códigos del libro están disponibles en
\href{https://github.com/jaalonso/Examenes_de_PF_con_Haskell_Vol11}
     {GitHub}
     \footnote{{\url{https://github.com/jaalonso/Examenes_de_PF_con_Haskell_Vol11}}}

Este libro es el 11º volumen de la serie de recopilaciones de
exámenes de programación funcional con Haskell. Los volúmenes anteriores son
\begin{itemize}
\item
  \href{https://github.com/jaalonso/Examenes_de_PF_con_Haskell_Vol1}
  {Exámenes de ``Programaci\'on funcional con Haskell''.
    Vol. 1 (Curso 2009--10)}\
    \footnote{\url{{https://github.com/jaalonso/Examenes_de_PF_con_Haskell_Vol1}}}

\item
  \href{https://github.com/jaalonso/Examenes_de_PF_con_Haskell_Vol2}
  {Exámenes de ``Programaci\'on funcional con Haskell''.
    Vol. 2 (Curso 2010--11)}\
    \footnote{\url{{https://github.com/jaalonso/Examenes_de_PF_con_Haskell_Vol2}}}

\item
  \href{https://github.com/jaalonso/Examenes_de_PF_con_Haskell_Vol3}
  {Exámenes de ``Programaci\'on funcional con Haskell''.
    Vol. 3 (Curso 2011--12)}\
    \footnote{\url{{https://github.com/jaalonso/Examenes_de_PF_con_Haskell_Vol3}}}

\item
  \href{https://github.com/jaalonso/Examenes_de_PF_con_Haskell_Vol4}
  {Exámenes de ``Programaci\'on funcional con Haskell''.
    Vol. 4 (Curso 2012--13)}\
    \footnote{\url{{https://github.com/jaalonso/Examenes_de_PF_con_Haskell_Vol4}}}

\item
  \href{https://github.com/jaalonso/Examenes_de_PF_con_Haskell_Vol5}
  {Exámenes de ``Programaci\'on funcional con Haskell''.
    Vol. 5 (Curso 2013--14)}\
    \footnote{\url{{https://github.com/jaalonso/Examenes_de_PF_con_Haskell_Vol5}}}

\item
  \href{https://github.com/jaalonso/Examenes_de_PF_con_Haskell_Vol6}
  {Exámenes de ``Programaci\'on funcional con Haskell''.
    Vol. 6 (Curso 2014--15)}\
    \footnote{\url{{https://github.com/jaalonso/Examenes_de_PF_con_Haskell_Vol6}}}

\item
  \href{https://github.com/jaalonso/Examenes_de_PF_con_Haskell_Vol7}
  {Exámenes de ``Programaci\'on funcional con Haskell''.
    Vol. 6 (Curso 2015--16)}\
    \footnote{\url{{https://github.com/jaalonso/Examenes_de_PF_con_Haskell_Vol7}}}

\item
  \href{https://github.com/jaalonso/Examenes_de_PF_con_Haskell_Vol8}
  {Exámenes de ``Programaci\'on funcional con Haskell''.
    Vol. 8 (Curso 2016--17)}\
    \footnote{\url{{https://github.com/jaalonso/Examenes_de_PF_con_Haskell_Vol8}}}

\item
  \href{https://github.com/jaalonso/Examenes_de_PF_con_Haskell_Vol9}
  {Exámenes de ``Programaci\'on funcional con Haskell''.
    Vol. 9 (Curso 2017--18)}\
    \footnote{\url{{https://github.com/jaalonso/Examenes_de_PF_con_Haskell_Vol9}}}

\item
  \href{https://github.com/jaalonso/Examenes_de_PF_con_Haskell_Vol10}
  {Exámenes de ``Programaci\'on funcional con Haskell''.
    Vol. 10 (Curso 2018--19)}\
    \footnote{\url{{https://github.com/jaalonso/Examenes_de_PF_con_Haskell_Vol10}}}

\end{itemize}
     
\begin{flushright}
  José A. Alonso \\
  Sevilla, 31 de octubre de 2019
\end{flushright}

\chapter{Exámenes del grupo 1}
\chapterauthor{Antonia M. Chávez}
\section{Examen 1 (11 de noviembre de 2019)}
\examen{../Examenes/Grupo_1/examen_1_11_nov.hs}
% \section{Examen 2 (16 de diciembre de 2019)}
% \examen{../Examenes/Grupo_1/examen_12_16_dic.hs}
% \section{Examen 3 (23 de enero de 2020)}
% \examen{../Examenes/Grupo_1/examen_3_23_ene.hs}
% \section{Examen 4 (16 de marzo de 2020}
% \examen{../Examenes/Grupo_1/examen_4_16_mar.hs}
% \section{Examen 5 (11 de mayo de 2020)}
% \examen{../Examenes/Grupo_1/examen_5_11_may.hs}
% \section{Examen 6 (15 de junio de 2020)}
% \examen{../Examenes/Grupo_1/examen_6_15_jun.hs}
% \section{Examen 7 (23 de junio de 2020)}
% \examen{../Examenes/Grupo_1/examen_7_23_jun.hs}
% \section{Examen 8 (02 de septiembre de 2020)}
% \examen{../Examenes/Grupo_1/examen_8_02_sep.hs}
        
\chapter{Exámenes del grupo 2}
\chapterauthor{María J. Hidalgo}
\section{Examen 1 (06 de noviembre de 2019)}
\examen{../Examenes/Grupo_2/examen_1_06_nov.hs}
% \section{Examen 2 (27 de noviembre de 2019)}
% \examen{../Examenes/Grupo_2/examen_2_27_nov.hs}
% \section{Examen 3 (23 de enero de 2020)}
% \examen{../Examenes/Grupo_2/examen_3_23_ene.hs}
% \section{Examen 4 (13 de marzo de 2020)}
% \examen{../Examenes/Grupo_2/examen_4_13_mar.hs}
% \section{Examen 5 (24 de abril de 2020)}
% \examen{../Examenes/Grupo_2/examen_5_24_abr.hs}
% \section{Examen 6 (15 de junio de 2020)}
% \examen{../Examenes/Grupo_2/examen_6_15_jun.hs}
% \section{Examen 7 (23 de junio de 2020)}
% \examen{../Examenes/Grupo_2/examen_7_23_jun.hs}
% \section{Examen 8 (02 de septiembre de 2020)}
% \examen{../Examenes/Grupo_2/examen_8_02_sep.hs}
    
\chapter{Exámenes del grupo 3}
\chapterauthor{Francisco J. Martín}
\section{Examen 1 (28 de octubre de 2019}
\examen{../Examenes/Grupo_3/examen_1_28_oct.hs}
% \section{Examen 2 (02 de diciembre de 2019}
% \examen{../Examenes/Grupo_3/examen_2_02_dic.hs}
% \section{Examen 3 (23 de enero de 2020)}
% \examen{../Examenes/Grupo_3/examen_3_23_ene.hs}
% \section{Examen 4 (16 de marzo de 2020}
% \examen{../Examenes/Grupo_3/examen_4_16_mar.hs}
% \section{Examen 5 (04 de mayo de 2020}
% \examen{../Examenes/Grupo_3/examen_5_04_may.hs}
% \section{Examen 6 (15 de junio de 2020)}
% \examen{../Examenes/Grupo_3/examen_6_15_jun.hs}
% \section{Examen 7 (23 de junio de 2020)}
% \examen{../Examenes/Grupo_3/examen_7_23_jun.hs}
% \section{Examen 8 (02 de septiembre de 2020)}
% \examen{../Examenes/Grupo_3/examen_8_02_sep.hs}

\chapter{Exámenes del grupo 4}
\chapterauthor{José A. Alonso}
\section{Examen 1 (30 de octubre de 2019)}
\examen{../Examenes/Grupo_4/examen_1_30_oct.hs}
% \section{Examen 2 (18 de diciembre de 2019)}
% \examen{../Examenes/Grupo_4/examen_2_18_dic.hs}
% \section{Examen 3 (23 de enero de 2020)}
% \examen{../Examenes/Grupo_4/examen_3_23_ene.hs}
% \section{Examen 4 (18 de marzo de 2020)}
% \examen{../Examenes/Grupo_4/examen_4_18_mar.hs}
% \section{Examen 5 (06 de mayo de 2020)}
% \examen{../Examenes/Grupo_4/examen_5_06_may.hs}
% \section{Examen 6 (15 de junio de 2020)}
% \examen{../Examenes/Grupo_4/examen_6_15_jun.hs}
% \section{Examen 7 (23 de junio de 2020)}
% \examen{../Examenes/Grupo_4/examen_7_23_jun.hs}
% \section{Examen 8 (02 de septiembre de 2020)}
% \examen{../Examenes/Grupo_4/examen_8_02_sep.hs}

\chapter{Exámenes del grupo 5}
\chapterauthor{Andrés Cordón y Miguel A. Martínez}
\section{Examen 1 (06 de noviembre de 2019)}
\examen{../Examenes/Grupo_5/examen_1_06_nov.hs}
% \section{Examen 2 (11 de diciembre de 2019)}
% \examen{../Examenes/Grupo_5/examen_2_11_dic.hs}
% \section{Examen 3 (23 de enero de 2020)}
% \examen{../Examenes/Grupo_5/examen_3_23_ene.hs}
% \section{Examen 4 (11 de marzo de 2020)}
% \examen{../Examenes/Grupo_5/examen_4_11_mar.hs}
% \section{Examen 5 (06 de mayo de 2020)}
% \examen{../Examenes/Grupo_5/examen_5_06_may.hs}
% \section{Examen 6 (15 de junio de 2020)}
% \examen{../Examenes/Grupo_5/examen_6_15_jun.hs}
% \section{Examen 7 (23 de junio de 2020)}
% \examen{../Examenes/Grupo_5/examen_7_23_jun.hs}
% \section{Examen 8 (02 de septiembre de 2020)}
% \examen{../Examenes/Grupo_5/examen_8_02_sep.hs}

\appendix % Apéndices

% A. Resumen de funciones Haskell
\include{Apendice/resumen_Haskell}

% B. Método de Pólya para la resolución de problemas
\include{Apendice/metodo_de_Polya}

%%%%%%%%%%%%%%%%%%%%%%%%%%%%%%%%%%%%%%%%%%%%%%%%%%%%%%%%%%%%%%%%%%%%%%%%%%%%%%%
%%  Bibliografía                                                            %%
%%%%%%%%%%%%%%%%%%%%%%%%%%%%%%%%%%%%%%%%%%%%%%%%%%%%%%%%%%%%%%%%%%%%%%%%%%%%%%%

% \nocite{Alonso-12a}
\nocite{Alonso-12b}
\nocite{Bird-99a}
\nocite{Cunningham-10a}
\nocite{Daume-06}
\nocite{Davie-92a}
\nocite{Doets-04a}
\nocite{Fokker-96}
\nocite{Hudak-00a}
\nocite{Hudak-12a}
\nocite{Hutton-07a}
\nocite{OSullivan-08a}
\nocite{Rabhi-99a}
\nocite{Polya-65a}
\nocite{Ruiz-04}
\nocite{Thompson-11a}

\addcontentsline{toc}{chapter}{Bibliografía}
\bibliographystyle{abbrv}
\bibliography{Examenes_de_PF_con_Haskell_Vol11}

\end{document}

%%% Local Variables: 
%%% mode: latex
%%% TeX-master: t
%%% End: 

